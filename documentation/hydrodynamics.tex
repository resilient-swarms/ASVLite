\section{ASV motion in waves} \label{ASV motion in waves}
This section presents a summary of the theory of ASV response motion in waves
and and is mainly based on \cite{lewis1988principles} and
\cite{bhattacharyya1978dynamics}.

ASV in waves experiences oscillatory motions and these motions have six degrees
of freedom that is three transitional components - surge (in longitudinal
direction, x), sway (in transverse direction, y) and heave (in vertical
direction, z), and three angular components - roll (about the longitudinal axis,
x), pitch (about the transverse axis, y) and yaw (about the vertical axis, z).
Of the six motions, only three are purely oscillatory in nature- heave, pitch
and roll. This is because these motions causes a change in the equilibrium
between gravitational force and buoyancy force acting on the vessel resulting in
a restoring force which brings the vessel back to the equilibrium position.
Surge, sway and yaw does not produce a restoring force and hence these motions
are not oscillatory in nature unless the exciting force itself changes direction
and brings it back to the initial state.

At first, this section explores motion in each degrees of freedom independently
of others. That is, for example, it is assumed that the heave motion is not
affect by and does not affect any other motion. In reality this is not true.
Since the ASV is longitudinally asymmetric, the heave motion will also induce
pitching motion. The relationship between each motion in each degrees of freedom
is explored in the section \todo[inline]{Link to section for coupled motion}.

\subsection{Equations of motion}

The equations of motions are based on Newton's second law of motion. For each
transitional motions components, this means that the force acting on the vessel
should be equal the product of mass and acceleration in that direction and for
each angular motion components it means that the moment acting on the vessel
equals the product of mass moment of inertia and angular acceleration. 

Since the focus of this section is vessel response motion due to waves, the
forces and moments considered are only the fluid forces and moments due to waves
acting on the vessel. The forces and moments can be subdivided into two types -
Froude-Krylov and diffraction excitation. Froude-Krylov excitation force and
moment is obtained by integrating the pressure due to wave on the wetted surface
area of the hull. Froude-Krylov does not consider the effects vessel on the
incident wave. On the other hand, diffraction excitation are forces and moments
due to modification of incident wave due to the presence of the vessel. In cases
where the length of the incident wave is longer than the vessel length, the
diffraction excitation forces and moments will be of small magnitude and hence
is often ignored. 

